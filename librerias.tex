%-------------------------------------------------------------------------------
%                                Paquetes extras                               %
%-------------------------------------------------------------------------------
\setcounter{secnumdepth}{3}
\setcounter{tocdepth}{3} 
\usepackage{amsthm}
\theoremstyle{definition}
\newtheorem{axiom}{Axioma} %[chapter]
\renewcommand\theaxiom{\Roman{axiom}}

\theoremstyle{definition}
\newtheorem{property}{Propiedad} %[chapter]

\theoremstyle{definition}
\newtheorem{corollary}{Corolario} %[chapter]

\theoremstyle{definition}
\newtheorem{theorem}{Teorema} %[chapter]

\usepackage{lipsum}												% Texto de ejemplo \lipsum[1-30]
\decimalpoint													% Punto decimal en lugar de coma
\spanishsignitems												% Viñetas en lugar de cuadros
\raggedbottom													% Eliminar molestos warnings

\usepackage{comment}												% Comentarios largos
\usepackage{pdfpages}											% Incluir portada echa en Inkscape
\usepackage{setspace}											% Interlineado
\usepackage{makecell}											% Para tablas
\usepackage{xcolor}												% Colores en tablas
\usepackage{colortbl}
\usepackage{multirow}
\usepackage{array}												% Necesario para algunas tablas
\usepackage[inline]{enumitem}										% Personalizar itemize
\usepackage{multicol}											% Item 2 columns

\definecolor{Red}{RGB}{255,191,191}								% Colores definidos por el usuario

\usepackage[nottoc]{tocbibind}									% Bibliografia en table of contents

\usepackage[figuresright]{rotating}								% Rotar figuras con caption
\usepackage{subcaption}											% Subfiguras
%-------------------------------------------------------------------------------
%                            Comandos matematicos                              %
%-------------------------------------------------------------------------------
\usepackage{steinmetz}											% Para representar fasores
\usepackage{bm}													% Bold math  \bm command
\newcommand{\binomb}[2]{\genfrac{[}{]}{0pt}{}{#1}{#2}}
%-------------------------------------------------------------------------------
%                        Paquetes para hipervinculos                           %
%-------------------------------------------------------------------------------
\usepackage[hidelinks]{hyperref}								% Añade los bookmarks y le quita la caja roja, \url{}
\urlstyle{same}
%-------------------------------------------------------------------------------
%                           Estilos de encabezados                             %
%-------------------------------------------------------------------------------
\usepackage{fancyhdr, blindtext}								% Libreria para encabezados

\renewcommand{\chaptermark}[1]{\markboth{#1}{}}					% Capitulos y secciones en minusculas
\renewcommand{\sectionmark}[1]{\markright{#1}}

\fancypagestyle{normalstyle}{%
  \fancyhf{}													% Reinicial estilos de header y footer
	\fancyhead[LE,RO]{\thepage}
	\fancyhead[LO]{\nouppercase{\rightmark}}
	\fancyhead[RE]{\nouppercase{\leftmark}}
	\renewcommand{\headrulewidth}{0.4pt}
	\renewcommand{\footrulewidth}{0pt}
	\setlength{\headheight}{14.62pt}
}

\fancypagestyle{Resumen}{%
  \fancyhf{}													% Reinicial estilos de header y footer
	\fancyhead[LE,RO]{\thepage}
	\fancyhead[LO]{\nouppercase{Resumen}}
	\fancyhead[RE]{\nouppercase{Resumen}}
	\renewcommand{\headrulewidth}{0.4pt}
	\renewcommand{\footrulewidth}{0pt}
	\setlength{\headheight}{14.62pt}
}
%-------------------------------------------------------------------------------
%                            Libreria de codigos                               %
%-------------------------------------------------------------------------------
% Paquetes necesarios
\usepackage{listings}
\usepackage{xcolor}

% Colores para tablas
\definecolor{Gray}{RGB}{230,230,230}
\definecolor{Red}{RGB}{255,191,191}

% Colores personalizados
\definecolor{verde}{rgb}{0,0.6,0}
\definecolor{gris}{RGB}{253, 253, 253}
\definecolor{grisfuerte}{RGB}{140, 140, 140}

% Deficion de lenguajes perzonalizados

% Estilos MATLAB
\lstdefinestyle{MATLAB}{
	language=MATLAB,
	basicstyle=\linespread{1}\tiny\fontfamily{pcr}\selectfont,
	backgroundcolor=\color{gris},
	frame=single,
	frameround=tttt,
	rulecolor=\color{black},
	commentstyle=\color{verde},
	keywordstyle=\color{blue}, %magenta
	stringstyle=\color{grisfuerte},                  
	captionpos=t,                    
	breaklines=true,                       
	breakatwhitespace=false,
	showspaces=false,                
	showstringspaces=false,
	showtabs=false,
	keepspaces=true,
	columns=flexible,
	tabsize=4,   
}

\lstdefinestyle{VHDL}{
	language=VHDL,
	basicstyle=\linespread{1}\tiny\fontfamily{pcr}\selectfont,
	backgroundcolor=\color{gris},
	frame=single,
	frameround=tttt,
	rulecolor=\color{black},
	commentstyle=\color{verde},
	keywordstyle=\color{blue}, %magenta
	stringstyle=\color{grisfuerte},                  
	captionpos=t,                    
	breaklines=true,                       
	breakatwhitespace=false,
	showspaces=false,                
	showstringspaces=false,
	showtabs=false,
	keepspaces=true,
	columns=flexible,
	tabsize=4,
	upquote=true,
}


\lstdefinestyle{VHDL_TEXT}{
	language=VHDL,
	basicstyle=\linespread{1}\footnotesize\fontfamily{pcr}\selectfont,
	backgroundcolor=\color{gris},
	frame=single,
	frameround=tttt,
	rulecolor=\color{black},
	commentstyle=\color{verde},
	keywordstyle=\color{blue}, %magenta
	stringstyle=\color{grisfuerte},                  
	captionpos=t,                    
	breaklines=true,                       
	breakatwhitespace=false,
	showspaces=false,                
	showstringspaces=false,
	showtabs=false,
	keepspaces=true,
	columns=flexible,
	tabsize=4,
	upquote=true,
}


\lstdefinestyle{C}{
	language=C,
	basicstyle=\linespread{1}\tiny\fontfamily{pcr}\selectfont,
	backgroundcolor=\color{gris},
	frame=single,
	frameround=tttt,
	rulecolor=\color{black},
	commentstyle=\color{verde},
	keywordstyle=\color{blue}, %magenta
	stringstyle=\color{grisfuerte},                  
	captionpos=t,                    
	breaklines=true,                       
	breakatwhitespace=false,
	showspaces=false,                
	showstringspaces=false,
	showtabs=false,
	keepspaces=true,
	columns=flexible,
	tabsize=4,
	upquote=true,    
}

\renewcommand{\lstlistingname}{Código}% Listing -> Algorithm
\renewcommand{\lstlistlistingname}{Lista de códigos}% 


%-------------------------------------------------------------------------------
%                           Caption en negritas                                %
%-------------------------------------------------------------------------------
\usepackage[labelfont=bf]{caption}
\captionsetup{labelfont=bf}


%-------------------------------------------------------------------------------
%                           Lista de conceptos                                %
%-------------------------------------------------------------------------------
\usepackage[nopostdot,style=super,nonumberlist,toc,nogroupskip]{glossaries}
\setlength{\glsdescwidth}{0.7\textwidth}
\makeglossaries

\newglossaryentry{FPGA}
{
    name=FPGA, %% $\qquad\qquad\qquad\qquad$
    description={Field Programmable Logic Array}
}

\newglossaryentry{RNG}
{
    name=RNG,
    description={Random Number Generator}
}

\newglossaryentry{TRNG}
{
    name=TRNG,
    description={True Random Number Generator}
}

\newglossaryentry{TERO}
{
    name=TERO,
    description={Transient Effect Ring Oscillator}
}

\newglossaryentry{ERO-TRNG}
{
    name=ERO-TRNG,
    description={Elementary ring oscillator based TRNG}
}

\newglossaryentry{COSO-TRNG}
{
    name=COSO-TRNG,
    description={Coherent sampling ring oscillator based TRNG}
}

\newglossaryentry{MURO-TRNG}
{
    name=MURO-TRNG,
    description={Multi-ring oscillator based TRNG}
}


\newglossaryentry{TERO-TRNG}
{
    name=TERO-TRNG,
    description={Transient effect ring oscillator based TRNG}
}

\newglossaryentry{PLL-TRNG}
{
    name=PLL-TRNG,
    description={Phase-locked loop based TRNG}
}


\newglossaryentry{STR-TRNG}
{
    name=STR-TRNG,
    description={Self-timed ring based TRNG}
}

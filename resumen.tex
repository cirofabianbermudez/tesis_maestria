\chapter{Resumen}
    
    En esta tesis se diseñó e implementó un TRNG híbrido en la FPGA Xilinx Artix 7 xc7a35tcpg236-1 sobre la tarjeta de desarrollo Digilent Basys 3. Se utilizó un núcleo ERO-TRNG para generar una semilla de 64 bits que funciona como condición inicial para un mapa caótico bidimensional. Haciendo uso de la operación mod 256 se extraen 16 bits aleatorios por cada iteración del mapa. 

    En este trabajo se presenta toda la teoría necesaria para comprender los generadores de números aleatorios, así como su clasificación, fuentes de aleatoriedad, parámetros de evaluación pruebas estadísticas y arquitecturas de núcleos TRNG específicas para FPGA. Se estudian brevemente las características principales de los mapas caóticos como puntos fijos, estabilidad lineal, diagramas de bifurcación y diagramas de cobwebs. Después se expone la metodología de diseño para implementar en FPGA el mapa caótico bidimensional y el núcleo ERO-TRNG utilizando el lenguaje de descripción de hardware VHDL. El mapa caótico se implementó utilizando aritmética de punto fijo de 64 bits, 3 bits para la parte entera, 60 bits para la fraccionaria y un bit de signo y para comprobar su funcionamiento se empleó un simulador desarrollado lenguaje C. Posteriormente se analizó el dominio de atracción del mapa para diferentes parámetros con el fin de poder seleccionar un rango en que las condiciones iniciales produzcan caos, por ultimó se utilizaron multiplicadores de una sola constante para reducir el uso de recursos. Utilizando diversos elementos digitales básicos y el núcleo ERO-TRNG se diseñó un generador de semillas de 64 bits que alimenta a las condiciones iniciales del mapa caótico. Finalmente, las secuencias binarias obtenidas por el TRNG híbrido se mandaron a una computadora utilizando el protocolo de comunicación RS232 y se analizaron con las pruebas estadísticas NIST SP 800-22.

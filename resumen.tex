\chapter{Resumen}
    
    En esta tesis se diseño un TRNG híbrido en una FPGA Xilinx Artix 7 xc7a35tcpg236-1 utilizando un ERO-TRNG y un mapa caótico bidimensional. Se presenta toda la teoría necesaria para comprender los generadores de números aleatorios así como su clasificación, fuentes de aleatoriedad, parámetros de evaluación y arquitecturas de núcleos TRNG específicas para FPGA. Se estudian brevemente las características principales de los mapas caóticos y se expone la metodología de diseño para implementar tanto el mapa caótico como el ERO-TRNG utilizando el lenguaje de descripción de hardware VHDL. El mapa caótico se implementó utilizando aritmética de punto fijo de 64 bits y para comprobar su funcionamiento se empleó un simulador en lenguaje C. Finalmente las secuencias binarias obtenidas por el TRNG híbrido se mandaron a una computadora utilizando el protocolo de comunicación RS232 y se analizaron con las pruebas estadísticas NIST SP 800-22. 

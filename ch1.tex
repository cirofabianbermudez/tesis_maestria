\chapter{Introducción}

     Los números aleatorios se utilizan en muchos ámbitos de nuestra vida. Los utilizamos para elegir quién gana la lotería, para determinar quién atacará primero en un partido de fútbol, para garantizar una partida justa en juegos de mesa y desempeñan un papel fundamental en la criptografía y la seguridad de la información. Para seleccionar al equipo atacante en un partido de fútbol, basta con lanzar una moneda. Sin embargo, para jugar a un juego de mesa se requieren más de dos valores aleatorios, por lo que utilizamos un dado. En cambio, la criptografía requiere algo más que tirar un dado para asegurar la protección  de nuestros datos en comunicaciones digitales o en transacciones bancarias. La seguridad de las comunicaciones es una parte fundamental de la vida moderna, en la que las personas envían correos electrónicos, realizan llamadas o envían mensajes a sus amigos y realiza transacciones en línea millones de veces al día. La sociedad confía que cada uno de estos procesos cotidianos sean seguros y confidenciales. La seguridad de las comunicaciones dependen de la capacidad de estos procesos para verificar la identidad de las personas que se comunican. La única forma de garantizar la seguridad es mediante la distribución de identidades privadas conocidas solo por el usuario, denominadas claves. Las claves privadas son números aleatorios únicos generados para cada usuario, que aseguran que personas malintencionadas no puedan suplantar a nadie y causar daño. La aleatoriedad de los números de las claves privadas es crucial para garantizar la seguridad de las conexiones. La capacidad de generar números aleatorios es, por tanto, una parte muy importante de la seguridad de los sistemas de comunicación.

    Los números aleatorios tienen una amplia variedad de aplicaciones. Se utilizan en simulaciones a computadora de fenómenos naturales, como en el modelado de la colisión de partículas en física nuclear y en el análisis logístico de pasajeros en aeropuertos en investigación operativa. También son útiles en el muestreo estadístico cuando no es práctico analizar todos los casos posibles. Los números aleatorios son una buena fuente de datos para probar la efectividad de los algoritmos de computadora y son cruciales para el funcionamiento de los algoritmos aleatorios. Además, se utilizan en análisis numérico y en la estética, donde agregar un poco de aleatoriedad hace que los gráficos generados por computadora y la música parezcan más vivos. En algunos casos, es importante tomar decisiones completamente imparciales y la aleatoriedad es esencial en las estrategias óptimas de la teoría de juegos. Ejecutivos y profesores universitarios recurren a estas estrategias con frecuencia, tirando una moneda o lanzando dardos. \cite{Knuth2014}. 

    En un sistema criptográfico, se utilizan generadores de números aleatorios o Random Number Generators (RNG), no sólo para generar claves criptográficas, sino también para generar números de un solo uso (nonces), valores de relleno, vectores de inicialización, desafíos y máscaras aleatorias para la protección contra ataques de canal lateral \cite{Petura2016}.

    A pesar de que hay muchas aplicaciones diferentes de los números aleatorios, todos comparten dos requisitos básicos: buenas propiedades estadísticas, concretamente una distribución de probabilidad uniforme donde cada valor de cualquier número aleatorio tenga la misma probabilidad de aparecer, e imprevisibilidad de los números aleatorios. Los números aleatorios, especialmente los utilizados para parámetros secretos como las claves, deben ser impredecibles para evitar que un atacante pueda calcular valores futuros o anteriores a partir de los datos ya generados y capturados. En los diseños modernos, se requieren algunas características adicionales: el generador debe ser intrínsecamente seguro, robusto y resistente a los ataques y probado en línea mediante pruebas específicas del generador \cite{Badrignans2011}. 

    En resumen, los números aleatorios son una herramienta valiosa y versátil que se utiliza en prácticamente todos los ámbitos de la ciencia y la tecnología y su importancia en la seguridad criptografía no puede ser subestimada.		
	
\cite{Jun1999}

	\section{Objetivos}
	
		\subsection{Objetivo general}
			\begin{itemize}
				\item Diseño e implementación en FPGA de un TRNG híbrido para la generación de secuencias muy largas y utilizarla para cifrar una imagen.
			\end{itemize}
		
		\subsection{Objetivos específicos}
			\begin{itemize}
                \item Investigar el estado del arte de diferentes generadores de números aleatorios y sus aplicaciones en cifradores.
                \item Estudiar los diferentes tipos de generadores de números aleatorios y analizar sus características principales.
                \item Estudiar la teoría de los mapas caóticos y su utilidad en generadores de números aleatorios.
                \item Diseñar un generador de números aleatorios híbrido utilizando un TRNG como generador de semillas y un mapa caótico para el postprocesamiento.
			\end{itemize}
